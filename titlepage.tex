

% ORCID icon macro
\newcommand{\orcidicon}[1]{%
    \href{https://orcid.org/#1}{\textcolor[HTML]{A6CE39}{\aiOrcid}}%
}

% --- Titlepage (place after \begin{document}) ---
\begin{titlepage}
    \onecolumn
    \centering
    \vspace*{2cm}

    {\LARGE\bfseries Research Software Engineering for Natural Sciences\par}

    \vspace{1.5cm}

    \setstretch{1.2}
    {\large
    Julian Dehne\textsuperscript{1}\;\orcidicon{0000-0001-9265-9619}\\
    Simon Christ\textsuperscript{2}\;\orcidicon{0000-0002-5866-1472}\\
    Florian Goth\textsuperscript{3}\;\orcidicon{0000-0003-2707-4790}\\
    Jean-Noël Grad\textsuperscript{4}\;\orcidicon{0000-0002-5821-4912}\\
    Magnus Hagdorn\textsuperscript{5}\;\orcidicon{0000-0002-5076-4864}\\
    Jan Philipp Thiele\textsuperscript{6}\;\orcidicon{0000-0002-2755-5087}\\
    Harald von Waldow\textsuperscript{7}\;\orcidicon{0000-0003-4800-2833}\\
    Jan Linxweiler\textsuperscript{8}\;\orcidicon{0000-0002-2755-5087}\\
    }

    \vspace{1.0cm}

    {\small
    \textsuperscript{1}\;RSE Working Group of German Computer Society, Berlin, Germany\\
    \textsuperscript{2}\;Department of Cell Biology and Biophysics, Leibniz University Hannover, Germany\\
    \textsuperscript{3}\;Würzburg-Dresden Cluster of Excellence ct.qmat, University of Würzburg, Germany\\
    \textsuperscript{4}\;Institute for Computational Physics, University of Stuttgart, Germany\\
    \textsuperscript{5}\;Geschäftsbereich IT, Charité Universitätsmedizin Berlin, Germany\\
    \textsuperscript{6}\;Technische Universität Braunschweig, Germany\\
    \textsuperscript{7}\;Johann Heinrich von Thünen Institute, Centre for Information Management, Germany\\
    \textsuperscript{8}\;Technische Universität Braunschweig, Germany\\
    }

    \vspace{2cm}

    % --- Abstract placeholder ---
    \begin{abstract}
        The role of Research Software Engineers (RSEs) has recently gained recognition, reflecting the growing importance of software in scientific research.
        While RSEs need expertise from software engineering, scientific computing, and open science, most lack formal training in software engineering.
        This disconnect limits the advances in the natural sciences, where computational work is often highly complex and needs access to state-of-the-art software engineering research.
        We argue that bridging this gap requires a dedicated curriculum rather than minor adjustments to existing programmes or continued education.
        A master’s programme in Research Software Engineering for the natural sciences can foster professionals with a dual identity:
        grounded in software engineering practices while deeply engaged with scientific research.
    \end{abstract}

    % --- Keywords placeholder ---
    \vspace{0.5cm}
    \textbf{Keywords:} Research Software Engineering, Natural Sciences, History, Identity, Curriculum

    \vfill
    {\small \today}

    \thispagestyle{empty}
    \clearpage
    \twocolumn
\end{titlepage}