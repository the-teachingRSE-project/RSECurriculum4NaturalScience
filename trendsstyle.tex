
% Die folgenden Definitionen sind i.w. identisch mit den
% Definitionen, die fuer Proceedings der IEEE vorgegeben werden.
% Papiere werden dort prinzipiell als article und zweispaltig
% gedruckt. Bei den IEEE-Definitionen werden Ueberschriften etwas
% kleiner und kompakter gesetzt (im Vergleich zu den
% Standard-Einstellungen, die sich mehr fuer 1-spaltigen Druck
% eignen. Wenn man das nicht mag, bleibt man bei den alten
% Definitionen. Bei den IEEE-Makros muss man unbequemerweise
% hinter jedem section-Kommando explizit \noindent angeben. 
% 
% Bei den IEEE-Definitionen muss leider hinter jeder
% Absatzueberschrift explizit ein \noindent stehen.

\makeatletter 
\def\@normalsize{\@setsize\normalsize{10pt}\xpt\@xpt
\abovedisplayskip 10pt plus2pt minus5pt\belowdisplayskip \abovedisplayskip
\abovedisplayshortskip \z@ plus3pt\belowdisplayshortskip 
6pt plus3pt minus3pt\let\@listi\@listI}
\def\subsize{\@setsize\subsize{12pt}\xipt\@xipt}
\def\section{\@startsection {section}{1}{\z@}
	{1.8ex plus 1ex minus .2ex} 
	{1.2ex plus .2ex \@afterindentfalse}
	{\large\bf}}
\def\subsection{\@startsection {subsection}{2}{\z@}
	{1.3ex plus 1ex}
	{.8ex plus .2ex \@afterindentfalse}
	{\subsize\bf}}
\def\paragraph{\@startsection {paragraph}{4}{\z@}
	{1.8ex plus .3ex}
	{-1em  \@afterindentfalse}
	{\normalsize\bf}}

\setlength{\textheight}{243mm}
\setlength{\columnsep}{6.5mm} %{2.0pc}
\setlength{\textwidth}{17cm}
\setlength{\parindent}{1pc}
\setlength{\parskip}{0.0cm}
\setlength{\topsep}{0.1cm}
\setlength{\partopsep}{0.0cm}
\setlength{\itemsep}{0.1cm}
\setlength{\parsep}{0.0cm}

% das folgende muss ggf. an die Einstellungen des 
% lokal vorhandenen Druckers angepasst werden.
\setlength{\topmargin}{-12mm}
\setlength{\oddsidemargin}{-6mm}
\setlength{\evensidemargin}{-6mm}

\pagestyle{empty}
\thispagestyle{empty}

\baselineskip12pt

